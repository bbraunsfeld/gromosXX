\documentclass[12pt,a4paper,english]{article} 

\voffset -1.5cm
\textheight 24cm
\hoffset -1.2cm
\textwidth 15.6cm

%%\usepackage[T1]{fontenc}
\usepackage[latin1]{inputenc} 
\usepackage{babel} 
\usepackage{setspace}

%% \usepackage{fancyheadings}
\usepackage{pslatex}
\usepackage{epsfig}
\usepackage{graphicx}
\usepackage{units}
\usepackage{booktabs}

%%\usepackage{cite}
\usepackage{citesort}  
\usepackage{overcite}
%%\usepackage{latexsym}

\usepackage{listings}

%%\doublespacing

\newcommand{\mbf}[1]{\mathbf{#1}}
\newcommand{\deri}[2]{\frac{\partial #2}{\partial #1}}
\renewcommand{\baselinestretch}{2.0}
\newcommand{\noun}[1]{\textsc{#1}}

\title{GromosXX}
\date{\today}
\author{Markus Christen${}^{\dagger}$
\thanks{Corresponding author (markus@igc.phys.chem.ethz.ch)}}

\begin{document}

\lstset{language=C++}

\maketitle
\noindent ${}^{\dagger}${Laboratory of Physical Chemistry, Swiss Federal Institute of
Technology Z\"urich, ETH-H\"onggerberg, CH-8093 Z\"urich, Switzerland}

\newpage

\begin{abstract}
We present the newest version of the Groningen Molecular Simulation
program package, GROMOS05. It has been developed for the dynamic
modelling of (bio)molecules using the methods of molecular dynamics,
stochastic dynamics, and energy minimization. An overview of GROMOS05 is
given, highlighting features not present in the last major release,
GROMOS96. The organization of the program package is outlined and the
included analysis package Gromos++ is described. Finally, three
applications illustrating the functionality are presented: gug, bla, and dong.
\newline
{\textbf{Keywords:} molecular dynamics simulation, programming, gromos
}
\end{abstract}

\newpage

\section{Introduction}
Starting with GROMOS87 a program package to facilitate research efforts
in the field of biomolecular simulation in a university environment has
been developed. After the last major release, GROMOS96, GROMOS has found
widespread use (hundreds of licences in over 40 countries on all
continents), helped by the fact that it has been designed for eaese of
extendability and that the complete source code is made available to
research establishments for a nominal fee. The program code has been
further developed in the group for computational chemistry at ETH
Z\"urich (Switzerland) throughout the recent years, leading now to a new
major release, GROMOS05. The enhancements were governed by following
criteria: (1) interest of our research group, (2) ease of use, (3)
extendability, (4) demonstrated usefulness or efficiency, (5)
well-defined and correct formulae and algorithms, and (6) computational
efficiency. The second criterium lead to a complete rewrite of the
analysis tools, now contained in the Gromos++ analysis package and the
third criterium lead to a complete rewrite of the MD engine into C++
while the reliable Fortran MD engine (promd) with many new features
features is still present in the current release.

In the next section, features of Gromos05 are described, highlighting
new functionality. Then the organization of the source
code is discussed. Section 4 gives an overview over the programs present
in the Gromos++ analysis package and finally in section 5 examples of
applications are shown.

\section{Features}

Here, the features of the MD engines will be listed. The enhanced
fortran MD engine (promd) retains all features of GROMOS96 and adds new
functionality. The C++ version contains most of the GROMOS96 features
(except 4D and path integral simulation) and a subset of the new
functionality. 

Features included are (non-complete):
\begin{itemize}
  \item Molecular Dynamics, Stochastic Dynamics simulation and energy
  \item Periodic boundary conditions (vacuum, rectangular, truncated
  octahedron, monoclinic and triclinic)
  minimiaation (steepest descent and conjugate gradient)
  \item Temperature control (weak coupling, Nose-Hoover, Nose-Hoover
  chains)
  \item Pressure control (weak coupling: isotropic, anisotropic and full
    anisotropic), separate coupling of any group of degrees of freedom
    for internal / rotational or translational temperature baths is possible
  \item Longrange interactions: straight cutoff, reaction field, ewald,
    particle-mesh-ewald, particle-partice-partice-mesh 
  \item Chargegroup based or atomic cutoff for nonbonded interactions
  \item Non-physical interaction: distance, dihedral, NOE and J-value
    restraints; distance constraints (SHAKE, M-SHAKE, LINCS)
  \item Relative free energy calculation, possibly including soft-core
    nonbonded interactions
  \item Enhanced sampling (Local elevation, replica exchange)
  \item Path integral simulation
  \item Grid based pairlist construction
  \item MPI and OMP parallelization
\end{itemize}

A complete description of the available features is included within the
manual.

\section{Algorithms}

\subsection{MD algorithm}

The complete MD algorithm based on the leap-frog scheme as implemented
by GROMOS is: 

\begin{enumerate}
\item Write position and velocity trajectory.
\item Remove centre of mass motion.
\item Initial positions and velocities are given, constraints satisfied.
\item Calculate (unconstrained) forces and energies from the potential energy
  function (using nearest image convention in case of periodic boundary
  conditions). At the same time calculate the molecular or atomic virial
  tensor.
\item Satisfy position constraints.
\item Update the velocities using the leap-frog scheme.
\item Apply temperature coupling (weak coupling or
  Nose-Hoover(-chains)).
\item Update the positions using the leap-frog scheme.
\item Satisfy distance constraints (using SHAKE, M-SHAKE or LINCS).
\item Calculate temperature(s).
\item Calculate pressure.
\item Apply pressure scaling (weak coupling).
\item Update lambda and topology for slow-growth simulations.
\item Calculate total energies, averages and fluctuations.
\item Print energies, write energy trajectories.
\end{enumerate}

This sequence is repeated for the required number of steps.

\subsection{Temperature coupling}

\subsection{Pressure coupling}

\subsection{Lattice sum methods}

\section{Implementation}

\subsection{promd (Fortran MD engine)}

The fortran MD engine promd is an enhancement of the GROMOS96 md
engine. It is written in Fortran77 with the exception of the use of
include files. An option to select different nonbonded routines has been
added. Macros to get rid of unwanted features (like 4D simulation or
unused periodicity code) have been added. Using these it is possible to
get a faster specialized build of promd. A cache-effective pairlist
generation method has been added. It additionally uses OpenMP to run in
parallel on shared memory multiprocessor systems.

\subsection{xxmd (C++ MD engine)}

The C++ MD engine (xxmd) has been rewritten from scratch. The major
motivation was to further increase the modularity and therefore the
extendability of the MD program. The code is split into two parts, the
first being an MD library containing basic functions necessary to run an
MD simulation, the second part the actual MD program. This second part
is very small. It is quite easy to add other specialized MD programs,
that make use of a subset of the functions provided in the library or
apply them in a different order. The source code of the library is again
split up into nine different parts: math, simulation, topology,
configuration, algorithm, interaction, io, util and check.

\begin{itemize}
  \item math contains classes for vectors, matrices and vector arrays,
    mathematical operations, physical constants and periodic boundary
    treatment.
  \item simulation contains the simulation parameters supplied to run
    an MD simulation.
  \item topology contains the topology of the simulated system, also
    includes possibly a perturbation topology.
  \item configuration contains the state of a system: its coordinates,
    velocities, forces, restraints data and so on.
  \item algorithm contains classes that use information from simulation
    and topology to act upon a configuration. All steps during an MD
    simulation can be described using an algorithm.
  \item interaction contains the biggest algorithm: the
    forcefield. Here, all interaction terms and their parameters are
    defined.
  \item io contains classes to read in or write out information. All
    file access is block oriented and human readable.
  \item util contains a few extra classes that are necessary to setup an
    MD simulation but don't belong exactly to it. Parsing command line
    arguments, generation of initial velocities or setting debug levels
    are examples of classes found herein.
  \item check contains test routines. Testing includes automatic calculation of
    energies under different conditions, then calculation of forces,
    energy lambda derivatives and virial and comparing to values
    obtained by finite difference calculations.
\end{itemize}

One step of an MD simulation consists of several algorithms applied to
the configuration in the right order. The Algorithm\_Sequence class is a
container for all the algorithms, they have to be inserted in the
correct order. Before the start of a simulation, all algorithms will be
initialized (by calling the init() function), during an MD step, the
algorithms are applied (apply()). The forcefield is also an algorithm,
which, when applied, calculates the forces and interaction energies of
all forcefield terms for the complete system. The forcefield terms
itself are Interaction classes. The forcefield is therefore a container
to store the different interaction objects. When the forcefield gets
applied, it calls calculate\_interactions() on all interaction
objects. These, in term, calculate the bonded interactions (bond length,
bond angles, improper dihedrals and dihedrals), the nonbonded
interactions (pairlist construction, long-range interactions,
short-range interactions and virial) and the non-physical forces
(distance-, dihedral-, NOE- or J-value-restraints).

The corresponding classes to the steps in the MD algorithm are shown in
{\em table \ref{tab:MD-classes}}.

\begin{table}[htbp]
\doublespacing
\begin{center}
\caption[Classes corresponding to MD algorithm steps.]{}
\label{tab:MD-classes}
\vspace{0.7cm}
\begin{tabular}{lll}
1. & Write position and velocity trajectory. & - \\
2. & Remove centre of mass motion. & Remove\_COM\_Motion \\
3. & Calculate (unconstrained) forces and energies & Forcefield \\
  & from the potential energ function & Bond\_Interaction \\
  & (using nearest image convention & Angle\_Interaction \\
  & in case of periodic boundary conditions). & Improper\_Dihedral\_Interaction \\
  & & Dihedral\_Interaction \\
  & & Nonbonded\_Interaction \\
  & & Position\_Restraints\_Interaction \\
  & & Distance\_Restraints\_Interaction \\
  & & NOE\_Restraints\_Interaction \\
  & & JValue\_Restraints\_Interaction \\
4. & Satisfy position constraints. & Position\_Constraints\_Interaction \\
5. & Update the velocities using the leap-frog scheme. &
Leapfrog\_Velocities \\
6. & Apply temperature coupling & Berendsen\_Thermostat \\
&  (weak coupling or Nose-Hoover(-chains)). & Nose\_Hoover\_Thermostat \\
7. & Update the positions using the leap-frog scheme. &
Leapfrog\_Positions \\
8. & Satisfy distance constraints & Shake \\
&  (using SHAKE, M-SHAKE or LINCS). & MShake \\
& & Lincs \\
9. & Calculate temperature(s). & Temperature\_Calculation \\
10. & Calculate pressure. & Pressure\_Calculation \\
11. & Apply pressure scaling (weak coupling). & Berendsen\_Barostat \\
12. & Update lambda and topology for slow-growth simulations. &
Slow\_Growth \\
13. & Calculate total energies, averages and fluctuations. &
Energy\_Calculation \\
14. & Print energies, write energy trajectories. & - \\
\end{tabular}
\end{center}
\end{table}

\subsubsection{Efficiency}

The main goal of writing a C++ MD engine was to further improve on
modularity (using some object-oriented features) and extendability
(using clear and common interfaces between the modules). Nevertheless,
an MD code has to be reasonably efficient to be of use. The complete
code is written in standard C++, no language extensions or machine
specific parts are used anywhere, resulting in a highly portable
program. This means that the compiler has to do all machine specific
optimizations. We believe that the easier maintenance of a single code
base and reasonably fast code with up-to-date compilers for the newest
hardware satisfy our needs best.

In the innerloops of the interaction calculation, templates are used to
generate specialized code. There are specialized periodicity classes for
the different types of periodic boundary conditions. The innerloop is
called with the boundary type as a template argument. Therefore the
compiler will generate a different, specialized version of the innerloop
for all boundary conditions (automatically). In the same manner, also
the interaction function term of the nonbonded interaction can be chosen
(switching between a Lennard-Jones type or a coarse-grained type
potential energy term) without the {\em if} statement showing up in the
compiled innerloop. An example code snippet is shown in {\em listing
  \ref{fig:templates}}.

%\begin{table}[htbp]
%\doublespacing
%\begin{center}
%\caption[Specialized code generation using templates.]{}
%\label{fig:templates}
%\vspace{0.7cm}
%\begin{figure}
\begin{lstlisting}[breaklines=true, caption={Specialized code generation
    using templates.}, frame=, label=fig:templates]
  enum boundary_type {vacuum, rectangular, triclinic};
  
  template<boundary_type boundary>
  class Periodicity;
  
  template<>
  class Periodicity<vacuum>{
    public:
      void nearest_image(Vec const & v1, Vec const & v2,
                         Vec & v3);
  };

  template<>
  class Periodicity<rectangular>{
    public:
      void nearest_image(Vec const & v1, Vec const & v2,
                         Vec & v3);
  };

  template<>
  class Periodicity<triclinic>{
    public:
      void nearest_image(Vec const & v1, Vec const & v2,
                         Vec & v3);
  };

  template<boundary_type boundary>
  class Interaction{
    public:
      virtual int calculate_interactions(
        Topology const & topology,
	Configuration & configuration,
	Simulation const & simulation){

	  Vec v;
	  Periodicity<boundary>
	    periodicity(configuration.current().box);

	  periodicity.nearest_image(
	    configuration.current().pos(0),
	    configuration.current().pos(1),
	    v);

	  // and so on

        }
  };

  int main(int argc, char **argv){

    Interaction<triclinic> interaction;
    interaction.calculate_interactions(
      topology, configuration, simulation);
      
    return 0;
  }

\end{lstlisting}
%\end{figure}
%\end{center}
%\end{table}


Some algorithms do rely on information from the previous step. To help
implementing those kind of algorithms, the complete current and old
state of the simulation is stored. During the leap-frog algorithm, the
current state changes to the old state and the updated information is
stored in the new current state. This is done by a simple (and very
fast) pointer exchange.

\subsubsection{Debug information}

It is often difficult to find where an error is exactly located. To help
in searching, a lot of debug information can be printed. Every debug
message is associated with a debug level. Only if the user requests a
high enough level, the message is printed. Additionally, every code
section belongs to a {\em module} and a {\em submodule}. Different debug
levels can be specified for all combinations. In that way, fine grained
control is achieved of how much information from which part of the MD
code should be printed.
 
\subsubsection{Pairlist generation}

Gromos uses a triple-range cutoff scheme for non-lattice-sum
calculations. A pairlist (chargegroup-based
or atomic) is calculated every N steps. Atompairs with a distance
smaller then the short-range cutoff are stored in a pairlist. These
interactions are calculated at every MD step. Pairs with a distance
between the short-range and the long-range cutoff are evaluated
directly, forces and energies stored and added to the short-range forces
and energies during the N-1 steps without pairlist creation. Outside the
long-range cutoff, a reaction field is applied.
 
Pairlist generation is a $O(N^2)$ operation. If a cutoff is used, it is
possible to have close to a $O(N)$ algorithm because every atom has only
a constant number of neighbours. The problem is to efficiently find
them. There are a number of grid-based pairlist algorithms
published. This implementation takes out one horizontal slab in the 3D
grid and extends it in all directions (using periodic copies of the
particles in the slab). Afterwards the interactions from all particles
above this slab within the cutoff (expressed as ranges of grid cells)
are calculated. In that way, no periodicity operations are necessary
anymore! For the short-range interactions a code can be stored in the
pairlist which shift vector to apply to the position (to get the correct
periodic copy), then no nearest image calls are necessary during the
nonbonded interaction calculation.

Using this information would make it possible to extract the virial
calculation from the innerloop of the nonbonded routines. This is
presently not implemented.

\subsubsection{Parallelization}

The interaction calculation is by far the computationally most expensive
part of an MD simulation. And then nonbonded interactions are the
responsible part of it. Again, the C++ code is focused on achieving
parallelization without comlicating the code. The nonbonded interaction
is split up into Nonbonded\_Sets, each containing its own storage space
for a pairlist, forces, energies and a virial tensor. While calculating
the pairlist, only a subset of atoms is considered, therefore each
processor creates its own partial pairlist. The interactions are
calculated from this partial pairlist and stored in the local arrays. In
the end the forces, energies and virial tensors of all nonbonded sets
are summed up and stored in the configuration. If the parallelization is
using distributed memory the positions have to be copied over from the
master to all other processes in the beginning as well as the reduction
of the energies, forces and virial in the end. In that case MPI is used
as communication protocol.

Effective parallelization could be achieved with only a few lines of
code (almost) completely separate from the nonbonded routines.

\subsubsection{Replica exchange}

The replica exchange method was implemented using MPI as communication
protocol. A server distributes the jobs to a dynamic number of
clients. After a given number of simulation steps they report back the
current energies, the server calculates switching probablilities and
prepares the next jobs for the clients. Replica exchange can be done in
temperature or in parameter ($\lambda$) space.

\section{Analysis: Gromos++}

All the analysis programs have been rewritten in C++. They accept a
standard set of command line arguments to specify input. It is easy to
add new analysis programs using the functionality provided within the
Gromos++ library. Following is a short description of the existing
programs:

\begin{itemize}
\item preparation
  \begin{itemize}
  \item probox puts a simulation box around a molecule and fills it with
    solvent molecules
  \item buildbox builds a simulation box containing N molecules at a
    specified density
  \item bin\_box builds a simulation box containing a binary mixture at
    a specified density
  \item ran\_box builds a simulation box containing N molecules at a
    specified density placing and orienting them randomly
  \item ran\_solvation builds a simulation box around a molecule and
    fills it randomly with solvent molecules
  \item checkbox checks box properties (size)
  \item copy\_box multiplies a box in any direction
  \item explode increase inter-molecule distance to vacuum conditions
  \item procry applies rotations and translations to a system to create
    a crystal unit cell
  \item maketop builds a topology from a building block sequence
  \item comtop combines two topologies
  \item contop converts topologies
  \item redtop reduces topologies by specified parts
  \item pt\_top combines topologies with perturbation topologies to
    produce new (perturbation) topologies
  \item perttopo creates a perturbation topology to perturb specified 
    atoms to dummies
  \item checktopo checks topologies for common mistakes
  \item progch generates or deletes hydrogen atom coordinates of a molecule
  \item proion replaces solvent molecules by an ion
  \item progca generates atomic Cartesian coordinates from a set of
    internal coordinates
  \item mkscript prepares an MD job script
  \end{itemize}
\item analyzation
  \begin{itemize}
  \item tstrip removes solvent from a trajectory
  \item filter filters out specified atoms from a trajectory
  \item cog calculates centre of geometries for specified atoms
  \item tser calculates time series of specified properties (distances,
    angles, torsions, order parameters, ...)
  \item dist calculates distributions of specified properties
  \item ditrans monitors dihedral angle transitions
  \item propertyrmsd calculates root-mean-square deviations over a set
    of properties
  \item dipole calculates dipole moments with respect to the centre of molecules
  \item epsilon calculates the dielectric constant for liquids
  \item rmsd calculates positional root-mean-square deviation between
    structures
  \item rmsf calculates positional room-mean-square fluctuations of
    specified atoms
  \item ene\_ana calculates averages, fluctuations and error estimates
    of energies, pressure and volume
  \item visco\_ana calculates the viscosity of liquids
  \item protcf calculates time correlation functions of time series
  \item rgyr calculates the radius of gyration
  \item rdf calculates the radial distribution function of selected atoms
  \item mdf gives the time series of the closest particle to a selected atom
  \item m\_widom performs particle insertion
  \item proarse Program to calculate the ARea 'Surface of' Estimate of a
    specified (part of a) molecule(s). Some (not so many actually)
    people also call this the Solvent-Accessible-Surface-Area (SASA).
  \item hbond analyses hydrogen bonding
  \item dssp analyses secondary structure elements
  \item prepnoe prepares for a NOE calculation
  \item noe calculates NOE distances
  \item postnoe analyses NOE distances
  \item oparam calculates order parameters for membranes
  \item nhoparam calculates N-H order parameters for membranes
  \item diffus calculates the diffusion coefficient
  \item neighbours prints a list of neighbouring atoms to a selected atom
  \item clusmol reads in the output of "neighbours" and determines
    clusters of molecules
  \item rmsdmat calculates the rmsd between all structure-pairs in a trajectory
  \item cluster analyses an rmsd-matrix to divide the structures into clusters
  \item postcluster analyses the cluster output for life-times, folding
    pathway and central-member structures
  \item iondens calculates ion densities 
  \item edyn does an essential dynamics analysis
  \item rot\_rel calculates the rotational relaxation time for solvent
  \item ener calculates any energy for a system 
  \item espmap calculates the vacuum electrostatic-potential on a grid
    around selected molecules of a given frame from the partial charges of
    the topology
  \end{itemize}
\item visualization / information
  \begin{itemize}
  \item g962pdb converts a Gromos topology and coordinates into pdb format
  \item frameout converts trajectories into other formats
  \item inbox puts the solute into the centre of the box
  \item atominfo prints (topological) information of specified atoms
  \item shake\_analysis analyses a specified configuration
  \item procmt\_list lists the atoms within a specified distance from a
    centre atom
  \end{itemize}
\end{itemize}

\section{Applications}

\section{Conclusions}

%\cite{Gunsteren:99.11}

\section{Acknowledgements}

Financial support by the National Center of Competence in Research (NCCR)
Structural Biology of the Swiss National Science Foundation (SNSF) is
gratefully acknowledged.

\begin{thebibliography}{10}

\bibitem{C++standard}
Programming languages - C++, ISO 14882, 2003.

\bibitem{AllenTildesley}
M. P. Allen and D. J. Tildesley, Computer Simulations of Liquids (Oxford
Science, Oxford, 1987).

\bibitem{Gunsteren:96.40}
  W.F. van Gunsteren, S.R. Billeter, A.A. Eising, P.H. H�nenberger, P. Kr�ger, 
  A.E. Mark, W.R.P. Scott, I.G. Tironi, Biomolecular Simulation: The GROMOS96
  Manual and User Guide, Vdf Hochschulverlag AG an der ETH Z�rich, Z�rich, Switzerland.

\bibitem{Gunsteren:99.11}
  W.R.P. Scott, P.H. H�nenberger, I.G. Tironi, A.E. Mark, S.R. Billeter,
  J. Fennen, A.E. Torda, T. Huber, P. Kr�ger and W.F. van Gunsteren, 
  J. Phys. Chem. A {\bf 103}, 3596 (1999).

\bibitem{Gunsteren:01.26}
  L.D. Schuler, X. Daura and W.F. van Gunsteren,
  J. Comput. Chem. {\bf 22}, 1205 (2001).

\end{thebibliography}

\pagebreak[4]

\renewcommand{\listtablename}{Table Captions}
\listoftables
\clearpage
\thispagestyle{empty}

\clearpage
\pagebreak[4] 

\renewcommand{\listfigurename}{Figure Captions}
\listoffigures

\clearpage
\pagebreak[4] 

\clearpage
\pagebreak[4] 

\clearpage
\end{document}
